% Options for packages loaded elsewhere
\PassOptionsToPackage{unicode}{hyperref}
\PassOptionsToPackage{hyphens}{url}
%
\documentclass[
]{book}
\usepackage{amsmath,amssymb}
\usepackage{iftex}
\ifPDFTeX
  \usepackage[T1]{fontenc}
  \usepackage[utf8]{inputenc}
  \usepackage{textcomp} % provide euro and other symbols
\else % if luatex or xetex
  \usepackage{unicode-math} % this also loads fontspec
  \defaultfontfeatures{Scale=MatchLowercase}
  \defaultfontfeatures[\rmfamily]{Ligatures=TeX,Scale=1}
\fi
\usepackage{lmodern}
\ifPDFTeX\else
  % xetex/luatex font selection
\fi
% Use upquote if available, for straight quotes in verbatim environments
\IfFileExists{upquote.sty}{\usepackage{upquote}}{}
\IfFileExists{microtype.sty}{% use microtype if available
  \usepackage[]{microtype}
  \UseMicrotypeSet[protrusion]{basicmath} % disable protrusion for tt fonts
}{}
\makeatletter
\@ifundefined{KOMAClassName}{% if non-KOMA class
  \IfFileExists{parskip.sty}{%
    \usepackage{parskip}
  }{% else
    \setlength{\parindent}{0pt}
    \setlength{\parskip}{6pt plus 2pt minus 1pt}}
}{% if KOMA class
  \KOMAoptions{parskip=half}}
\makeatother
\usepackage{xcolor}
\usepackage{color}
\usepackage{fancyvrb}
\newcommand{\VerbBar}{|}
\newcommand{\VERB}{\Verb[commandchars=\\\{\}]}
\DefineVerbatimEnvironment{Highlighting}{Verbatim}{commandchars=\\\{\}}
% Add ',fontsize=\small' for more characters per line
\usepackage{framed}
\definecolor{shadecolor}{RGB}{248,248,248}
\newenvironment{Shaded}{\begin{snugshade}}{\end{snugshade}}
\newcommand{\AlertTok}[1]{\textcolor[rgb]{0.94,0.16,0.16}{#1}}
\newcommand{\AnnotationTok}[1]{\textcolor[rgb]{0.56,0.35,0.01}{\textbf{\textit{#1}}}}
\newcommand{\AttributeTok}[1]{\textcolor[rgb]{0.13,0.29,0.53}{#1}}
\newcommand{\BaseNTok}[1]{\textcolor[rgb]{0.00,0.00,0.81}{#1}}
\newcommand{\BuiltInTok}[1]{#1}
\newcommand{\CharTok}[1]{\textcolor[rgb]{0.31,0.60,0.02}{#1}}
\newcommand{\CommentTok}[1]{\textcolor[rgb]{0.56,0.35,0.01}{\textit{#1}}}
\newcommand{\CommentVarTok}[1]{\textcolor[rgb]{0.56,0.35,0.01}{\textbf{\textit{#1}}}}
\newcommand{\ConstantTok}[1]{\textcolor[rgb]{0.56,0.35,0.01}{#1}}
\newcommand{\ControlFlowTok}[1]{\textcolor[rgb]{0.13,0.29,0.53}{\textbf{#1}}}
\newcommand{\DataTypeTok}[1]{\textcolor[rgb]{0.13,0.29,0.53}{#1}}
\newcommand{\DecValTok}[1]{\textcolor[rgb]{0.00,0.00,0.81}{#1}}
\newcommand{\DocumentationTok}[1]{\textcolor[rgb]{0.56,0.35,0.01}{\textbf{\textit{#1}}}}
\newcommand{\ErrorTok}[1]{\textcolor[rgb]{0.64,0.00,0.00}{\textbf{#1}}}
\newcommand{\ExtensionTok}[1]{#1}
\newcommand{\FloatTok}[1]{\textcolor[rgb]{0.00,0.00,0.81}{#1}}
\newcommand{\FunctionTok}[1]{\textcolor[rgb]{0.13,0.29,0.53}{\textbf{#1}}}
\newcommand{\ImportTok}[1]{#1}
\newcommand{\InformationTok}[1]{\textcolor[rgb]{0.56,0.35,0.01}{\textbf{\textit{#1}}}}
\newcommand{\KeywordTok}[1]{\textcolor[rgb]{0.13,0.29,0.53}{\textbf{#1}}}
\newcommand{\NormalTok}[1]{#1}
\newcommand{\OperatorTok}[1]{\textcolor[rgb]{0.81,0.36,0.00}{\textbf{#1}}}
\newcommand{\OtherTok}[1]{\textcolor[rgb]{0.56,0.35,0.01}{#1}}
\newcommand{\PreprocessorTok}[1]{\textcolor[rgb]{0.56,0.35,0.01}{\textit{#1}}}
\newcommand{\RegionMarkerTok}[1]{#1}
\newcommand{\SpecialCharTok}[1]{\textcolor[rgb]{0.81,0.36,0.00}{\textbf{#1}}}
\newcommand{\SpecialStringTok}[1]{\textcolor[rgb]{0.31,0.60,0.02}{#1}}
\newcommand{\StringTok}[1]{\textcolor[rgb]{0.31,0.60,0.02}{#1}}
\newcommand{\VariableTok}[1]{\textcolor[rgb]{0.00,0.00,0.00}{#1}}
\newcommand{\VerbatimStringTok}[1]{\textcolor[rgb]{0.31,0.60,0.02}{#1}}
\newcommand{\WarningTok}[1]{\textcolor[rgb]{0.56,0.35,0.01}{\textbf{\textit{#1}}}}
\usepackage{longtable,booktabs,array}
\usepackage{calc} % for calculating minipage widths
% Correct order of tables after \paragraph or \subparagraph
\usepackage{etoolbox}
\makeatletter
\patchcmd\longtable{\par}{\if@noskipsec\mbox{}\fi\par}{}{}
\makeatother
% Allow footnotes in longtable head/foot
\IfFileExists{footnotehyper.sty}{\usepackage{footnotehyper}}{\usepackage{footnote}}
\makesavenoteenv{longtable}
\usepackage{graphicx}
\makeatletter
\def\maxwidth{\ifdim\Gin@nat@width>\linewidth\linewidth\else\Gin@nat@width\fi}
\def\maxheight{\ifdim\Gin@nat@height>\textheight\textheight\else\Gin@nat@height\fi}
\makeatother
% Scale images if necessary, so that they will not overflow the page
% margins by default, and it is still possible to overwrite the defaults
% using explicit options in \includegraphics[width, height, ...]{}
\setkeys{Gin}{width=\maxwidth,height=\maxheight,keepaspectratio}
% Set default figure placement to htbp
\makeatletter
\def\fps@figure{htbp}
\makeatother
\setlength{\emergencystretch}{3em} % prevent overfull lines
\providecommand{\tightlist}{%
  \setlength{\itemsep}{0pt}\setlength{\parskip}{0pt}}
\setcounter{secnumdepth}{5}
\usepackage{booktabs}
\usepackage{amsmath}
\ifLuaTeX
  \usepackage{selnolig}  % disable illegal ligatures
\fi
\usepackage[]{natbib}
\bibliographystyle{plainnat}
\usepackage{bookmark}
\IfFileExists{xurl.sty}{\usepackage{xurl}}{} % add URL line breaks if available
\urlstyle{same}
\hypersetup{
  pdftitle={UNIP - Matemática e Estatística - Anotações de Aula},
  pdfauthor={Professor Miguel Suez Xve Penteado; },
  hidelinks,
  pdfcreator={LaTeX via pandoc}}

\title{UNIP - Matemática e Estatística - Anotações de Aula}
\usepackage{etoolbox}
\makeatletter
\providecommand{\subtitle}[1]{% add subtitle to \maketitle
  \apptocmd{\@title}{\par {\large #1 \par}}{}{}
}
\makeatother
\subtitle{Curso de Análise de Sistemas - 2025}
\author{Professor Miguel Suez Xve Penteado \and }
\date{2025-02-17}

\usepackage{amsthm}
\newtheorem{theorem}{Theorem}[chapter]
\newtheorem{lemma}{Lemma}[chapter]
\newtheorem{corollary}{Corollary}[chapter]
\newtheorem{proposition}{Proposition}[chapter]
\newtheorem{conjecture}{Conjecture}[chapter]
\theoremstyle{definition}
\newtheorem{definition}{Definition}[chapter]
\theoremstyle{definition}
\newtheorem{example}{Example}[chapter]
\theoremstyle{definition}
\newtheorem{exercise}{Exercise}[chapter]
\theoremstyle{definition}
\newtheorem{hypothesis}{Hypothesis}[chapter]
\theoremstyle{remark}
\newtheorem*{remark}{Remark}
\newtheorem*{solution}{Solution}
\begin{document}
\maketitle

{
\setcounter{tocdepth}{1}
\tableofcontents
}
\chapter*{Tópicos}\label{tuxf3picos}
\addcontentsline{toc}{chapter}{Tópicos}

This is a \emph{sample} book written in \textbf{Markdown}. You can use anything that Pandoc's Markdown supports; for example, a math equation \(a^2 + b^2 = c^2\).

\section{Usage}\label{usage}

Each \textbf{bookdown} chapter is an .Rmd file, and each .Rmd file can contain one (and only one) chapter. A chapter \emph{must} start with a first-level heading: \texttt{\#\ A\ good\ chapter}, and can contain one (and only one) first-level heading.

Use second-level and higher headings within chapters like: \texttt{\#\#\ A\ short\ section} or \texttt{\#\#\#\ An\ even\ shorter\ section}.

The \texttt{index.Rmd} file is required, and is also your first book chapter. It will be the homepage when you render the book.

\section{Render book}\label{render-book}

You can render the HTML version of this example book without changing anything:

\begin{enumerate}
\def\labelenumi{\arabic{enumi}.}
\item
  Find the \textbf{Build} pane in the RStudio IDE, and
\item
  Click on \textbf{Build Book}, then select your output format, or select ``All formats'' if you'd like to use multiple formats from the same book source files.
\end{enumerate}

Or build the book from the R console:

\begin{Shaded}
\begin{Highlighting}[]
\NormalTok{bookdown}\SpecialCharTok{::}\FunctionTok{render\_book}\NormalTok{()}
\end{Highlighting}
\end{Shaded}

To render this example to PDF as a \texttt{bookdown::pdf\_book}, you'll need to install XeLaTeX. You are recommended to install TinyTeX (which includes XeLaTeX): \url{https://yihui.org/tinytex/}.

\section{Preview book}\label{preview-book}

As you work, you may start a local server to live preview this HTML book. This preview will update as you edit the book when you save individual .Rmd files. You can start the server in a work session by using the RStudio add-in ``Preview book'', or from the R console:

\begin{Shaded}
\begin{Highlighting}[]
\NormalTok{bookdown}\SpecialCharTok{::}\FunctionTok{serve\_book}\NormalTok{()}
\end{Highlighting}
\end{Shaded}

\chapter*{Introdução}\label{introduuxe7uxe3o}
\addcontentsline{toc}{chapter}{Introdução}

Embora a matemática acompanhe a humanidade desde os primórdios da civilização,
o advento da computação levou o emprego da matemática a um nível de sem precendente
na história graças ao aparecimento do Software. Ela é a linguagem universal que criou o mundo digital.
É a ferramenta que permite aos programadores dar vida aos \textbf{algoritmos}, as \emph{receitas que ditam o comportamento dos computadores}.

Todas as linguagens de programação, em especial as fortemente tipadas, como \texttt{FORTRAN},
\texttt{C}, \texttt{JAVA}, \texttt{PASCAL}, \texttt{Python} e mesmo \texttt{SQL} usam conceitos matemáticos para
criar estruturas abstratas dentro dos computadores, os chamados \textbf{SOFTWARES}.

Esta disciplina tem o objetivo de demonstrar a utilidade dos conceitos matemáticos,
já conhecidos em sua maioria, no mundo da computação.

\section{Expressões Algébricas}\label{expressuxf5es-alguxe9bricas}

All chapter sections start with a second-level (\texttt{\#\#}) or higher heading followed by your section title, like the sections above and below here. You can have as many as you want within a chapter.

\subsection*{An unnumbered section}\label{an-unnumbered-section}
\addcontentsline{toc}{subsection}{An unnumbered section}

Chapters and sections are numbered by default. To un-number a heading, add a \texttt{\{.unnumbered\}} or the shorter \texttt{\{-\}} at the end of the heading, like in this section.

\chapter{Agenda das Aulas}\label{agenda-das-aulas}

\textbf{Fevereiro 2025}

\begin{longtable}[]{@{}
  >{\centering\arraybackslash}p{(\columnwidth - 6\tabcolsep) * \real{0.2500}}
  >{\centering\arraybackslash}p{(\columnwidth - 6\tabcolsep) * \real{0.2500}}
  >{\centering\arraybackslash}p{(\columnwidth - 6\tabcolsep) * \real{0.2500}}
  >{\centering\arraybackslash}p{(\columnwidth - 6\tabcolsep) * \real{0.2500}}@{}}
\toprule\noalign{}
\begin{minipage}[b]{\linewidth}\centering
No.
\end{minipage} & \begin{minipage}[b]{\linewidth}\centering
fevereiro 2025
\end{minipage} & \begin{minipage}[b]{\linewidth}\centering
Semana
\end{minipage} & \begin{minipage}[b]{\linewidth}\centering
conteúdo
\end{minipage} \\
\midrule\noalign{}
\endhead
\bottomrule\noalign{}
\endlastfoot
01 & 17/02/2025 & Segunda-feira & Expressões algébricas e proporção: conceitos e aplicações. \\
02 & 24/02/2025 & Segunda-feira & \\
\end{longtable}

\textbf{Março 2025}

\begin{longtable}[]{@{}cccc@{}}
\toprule\noalign{}
No. & Março 2025 & Semana & conteúdo \\
\midrule\noalign{}
\endhead
\bottomrule\noalign{}
\endlastfoot
03 & 03/03/2025 & Segunda-feira & \\
04 & 10/03/2025 & Segunda-feira & \\
05 & 17/03/2025 & Segunda-feira & \\
06 & 24/03/2025 & Segunda-feira & \\
07 & 31/03/2025 & Segunda-feira & \\
\end{longtable}

\textbf{Abril 2025}

\begin{longtable}[]{@{}cccc@{}}
\toprule\noalign{}
No. & Março 2025 & Semana & conteúdo \\
\midrule\noalign{}
\endhead
\bottomrule\noalign{}
\endlastfoot
08 & 07/04/2025 & Segunda-feira & \\
09 & 14/04/2025 & Segunda-feira & \\
10 & 21/04/2025 & Segunda-feira & \\
11 & 28/04/2025 & Segunda-feira & \\
\end{longtable}

\textbf{maio 2025}

\begin{longtable}[]{@{}cccc@{}}
\toprule\noalign{}
No. & Maio 2025 & Semana & conteúdo \\
\midrule\noalign{}
\endhead
\bottomrule\noalign{}
\endlastfoot
12 & 05/05/2025 & Segunda-feira & \\
13 & 12/05/2025 & Segunda-feira & \\
14 & 19/05/2025 & Segunda-feira & \\
15 & 26/05/2025 & Segunda-feira & \\
\end{longtable}

\textbf{junho 2025}

\begin{longtable}[]{@{}cccc@{}}
\toprule\noalign{}
No. & Junho 2025 & Semana & conteúdo \\
\midrule\noalign{}
\endhead
\bottomrule\noalign{}
\endlastfoot
12 & 02/06/2025 & Segunda-feira & \\
13 & 09/06/2025 & Segunda-feira & \\
14 & 16/06/2025 & Segunda-feira & \\
15 & 23/06/2025 & Segunda-feira & \\
\end{longtable}

\chapter{Matemática Básica - Revisão}\label{matemuxe1tica-buxe1sica---revisuxe3o}

Cross-references make it easier for your readers to find and link to elements in your book.

\section{Expresões Algébricas}\label{expresuxf5es-alguxe9bricas}

São denominadas \textbf{expressões algébricas} as expressões matemáticas que possuem apenas letras ou números e letras.

---------------------------------------------------------------------------------------------------------------------------------------

\begin{enumerate}
\def\labelenumi{\arabic{enumi})}
\item
  \(4x^2 - 10x\)
\item
  \(\frac{5x}{2}\)
\item
  \(2\sqrt{2x}+2\)
\end{enumerate}

---------------------------------------------------------------------------------------------------------------------------------------

\subsection{Monômios:}\label{monuxf4mios}

A expressão algébrica racional inteira é formada de apenas 1 termo:

\begin{longtable}[]{@{}
  >{\raggedright\arraybackslash}p{(\columnwidth - 0\tabcolsep) * \real{0.3889}}@{}}
\toprule\noalign{}
\endhead
\bottomrule\noalign{}
\endlastfoot
\begin{minipage}[t]{\linewidth}\raggedright
\begin{itemize}
\item
  \(z^2\)
\item
  \(4ac\)
\item
  \(7\sqrt{\frac{3}{4}}\)
\end{itemize}
\end{minipage} \\
\end{longtable}

\subsection{Polinômios:}\label{polinuxf4mios}

A expressão algébrica racional e inteira formada de 2 ou mais termos:

\begin{longtable}[]{@{}
  >{\raggedright\arraybackslash}p{(\columnwidth - 0\tabcolsep) * \real{0.9722}}@{}}
\toprule\noalign{}
\endhead
\bottomrule\noalign{}
\endlastfoot
\begin{minipage}[t]{\linewidth}\raggedright
\begin{itemize}
\item
  \(a^2 + b^2 = c^2\) (teorema de Pitágoras)
\item
  \(ax^2 + bx + c\) (equação da parábola)
\item
  \(x = -b \pm \frac{\sqrt{{b^2 -4ac}}}{2a}\) (fórmula de baskhara)
\item
  \(y=ax + b\) (equação da reta)
\end{itemize}
\end{minipage} \\
\end{longtable}

\subsection{Operações de SOMA e SUBTRAÇÃO com Polinômios:}\label{operauxe7uxf5es-de-soma-e-subtrauxe7uxe3o-com-polinuxf4mios}

Considere o polinômio a seguir:

\(4x + 3y + 9x + 7z + 3x - 8y + 12x - 1y\)

É possível simplifica-lo somando os monômios semelhantes \textbf{somando os coeficientes} e \textbf{preservando as partes literais}:

\begin{longtable}[]{@{}
  >{\raggedright\arraybackslash}p{(\columnwidth - 0\tabcolsep) * \real{1.0000}}@{}}
\toprule\noalign{}
\endhead
\bottomrule\noalign{}
\endlastfoot
\(                                         
  4x + 3y + 9x + 7z + 3x - 8y +12x -1y \\    
  4x + 9x + 3x + 12x + 3y -8y -1y + 7z \\    
  x(4 + 9 + 3 + 12) + y(3 - 8 -1) + z(7) \\  
  x(28) + y(-6) + z(7) \\                    
  28x -6y +7z                                
  \) \\
\end{longtable}

\subsection{Operações de MULTIPLICAÇÃO e DIVISÃO com Polinômios:}\label{operauxe7uxf5es-de-multiplicauxe7uxe3o-e-divisuxe3o-com-polinuxf4mios}

Considere o polinômio a seguir:

\(\frac{7m^3ng^2 * 32m * 40ng^4}{14m^2g^3}\)

É possível simplifica-lo multiplicando os monômios semelhantes \textbf{multiplicando (ou dividindo) os coeficientes} e \textbf{multiplicando (ou dividindo) as partes literais}:

\begin{longtable}[]{@{}
  >{\raggedright\arraybackslash}p{(\columnwidth - 0\tabcolsep) * \real{1.0000}}@{}}
\toprule\noalign{}
\endhead
\bottomrule\noalign{}
\endlastfoot
\(
\frac{7m^3ng^2 * 32m * 40ng^4}{14m^2g^3}  \\\frac{(7 * 32 * 40)( m^3* n * g^2 * m * n * g^4 )}{ (14)(m^2 * g^3) } \\\frac{(8.960)( m^3* m * n * n * g^2  *  g^4 )}{ (14)(m^2 * g^3) } \\\frac{(8.960)( m^4* n^2 * g^6   )}{ (14)(m^2 * g^3) } \\(640)( m^2 * n^2 * g^3   ) \\640m^2g^3
\) \\
\end{longtable}

Considere o polinômio a seguir:

\((5x - \frac{1}{2})\frac{3x}{4}\)

\begin{longtable}[]{@{}
  >{\raggedright\arraybackslash}p{(\columnwidth - 0\tabcolsep) * \real{1.0000}}@{}}
\toprule\noalign{}
\endhead
\bottomrule\noalign{}
\endlastfoot
\(                                        
 (5x - \frac{1}{2})\frac{3x}{4} \\          
 (\frac{5x*3x}{1*4} - \frac{1*3x}{2*4}) \\  
 (\frac{15x^2}{4}-\frac{3x}{8})             
 \) \\
\end{longtable}

\section{Porcentagem e Regra de Três}\label{porcentagem-e-regra-de-truxeas}

\chapter{Teoria dos Conjuntos}\label{teoria-dos-conjuntos}

You can add parts to organize one or more book chapters together. Parts can be inserted at the top of an .Rmd file, before the first-level chapter heading in that same file.

Add a numbered part: \texttt{\#\ (PART)\ Act\ one\ \{-\}} (followed by \texttt{\#\ A\ chapter})

Add an unnumbered part: \texttt{\#\ (PART\textbackslash{}*)\ Act\ one\ \{-\}} (followed by \texttt{\#\ A\ chapter})

Add an appendix as a special kind of un-numbered part: \texttt{\#\ (APPENDIX)\ Other\ stuff\ \{-\}} (followed by \texttt{\#\ A\ chapter}). Chapters in an appendix are prepended with letters instead of numbers.

\chapter{Funções AFIN}\label{funuxe7uxf5es-afin}

\section{Footnotes}\label{footnotes}

Footnotes are put inside the square brackets after a caret \texttt{\^{}{[}{]}}. Like this one \footnote{This is a footnote.}.

\section{Citations}\label{citations}

Reference items in your bibliography file(s) using \texttt{@key}.

For example, we are using the \textbf{bookdown} package \citep{R-bookdown} (check out the last code chunk in index.Rmd to see how this citation key was added) in this sample book, which was built on top of R Markdown and \textbf{knitr} \citep{xie2015} (this citation was added manually in an external file book.bib).
Note that the \texttt{.bib} files need to be listed in the index.Rmd with the YAML \texttt{bibliography} key.

The RStudio Visual Markdown Editor can also make it easier to insert citations: \url{https://rstudio.github.io/visual-markdown-editing/\#/citations}

\chapter{Funções Quadráticas}\label{funuxe7uxf5es-quadruxe1ticas}

\section{Equations}\label{equations}

Here is an equation.

\begin{equation} 
  f\left(k\right) = \binom{n}{k} p^k\left(1-p\right)^{n-k}
  \label{eq:binom}
\end{equation}

You may refer to using \texttt{\textbackslash{}@ref(eq:binom)}, like see Equation \eqref{eq:binom}.

\section{Theorems and proofs}\label{theorems-and-proofs}

Labeled theorems can be referenced in text using \texttt{\textbackslash{}@ref(thm:tri)}, for example, check out this smart theorem \ref{thm:tri}.

\begin{theorem}
\protect\hypertarget{thm:tri}{}\label{thm:tri}For a right triangle, if \(c\) denotes the \emph{length} of the hypotenuse
and \(a\) and \(b\) denote the lengths of the \textbf{other} two sides, we have
\[a^2 + b^2 = c^2\]
\end{theorem}

Read more here \url{https://bookdown.org/yihui/bookdown/markdown-extensions-by-bookdown.html}.

\section{Callout blocks}\label{callout-blocks}

The R Markdown Cookbook provides more help on how to use custom blocks to design your own callouts: \url{https://bookdown.org/yihui/rmarkdown-cookbook/custom-blocks.html}

\chapter{Exponenciais e Logarítmos}\label{exponenciais-e-logaruxedtmos}

\section{Publishing}\label{publishing}

HTML books can be published online, see: \url{https://bookdown.org/yihui/bookdown/publishing.html}

\section{404 pages}\label{pages}

By default, users will be directed to a 404 page if they try to access a webpage that cannot be found. If you'd like to customize your 404 page instead of using the default, you may add either a \texttt{\_404.Rmd} or \texttt{\_404.md} file to your project root and use code and/or Markdown syntax.

\section{Metadata for sharing}\label{metadata-for-sharing}

Bookdown HTML books will provide HTML metadata for social sharing on platforms like Twitter, Facebook, and LinkedIn, using information you provide in the \texttt{index.Rmd} YAML. To setup, set the \texttt{url} for your book and the path to your \texttt{cover-image} file. Your book's \texttt{title} and \texttt{description} are also used.

This \texttt{gitbook} uses the same social sharing data across all chapters in your book- all links shared will look the same.

Specify your book's source repository on GitHub using the \texttt{edit} key under the configuration options in the \texttt{\_output.yml} file, which allows users to suggest an edit by linking to a chapter's source file.

Read more about the features of this output format here:

\url{https://pkgs.rstudio.com/bookdown/reference/gitbook.html}

Or use:

\begin{Shaded}
\begin{Highlighting}[]
\NormalTok{?bookdown}\SpecialCharTok{::}\NormalTok{gitbook}
\end{Highlighting}
\end{Shaded}

\chapter{Matrizes e Determinantes}\label{matrizes-e-determinantes}

\section{Publishing}\label{publishing-1}

HTML books can be published online, see: \url{https://bookdown.org/yihui/bookdown/publishing.html}

\section{404 pages}\label{pages-1}

By default, users will be directed to a 404 page if they try to access a webpage that cannot be found. If you'd like to customize your 404 page instead of using the default, you may add either a \texttt{\_404.Rmd} or \texttt{\_404.md} file to your project root and use code and/or Markdown syntax.

\section{Metadata for sharing}\label{metadata-for-sharing-1}

Bookdown HTML books will provide HTML metadata for social sharing on platforms like Twitter, Facebook, and LinkedIn, using information you provide in the \texttt{index.Rmd} YAML. To setup, set the \texttt{url} for your book and the path to your \texttt{cover-image} file. Your book's \texttt{title} and \texttt{description} are also used.

This \texttt{gitbook} uses the same social sharing data across all chapters in your book- all links shared will look the same.

Specify your book's source repository on GitHub using the \texttt{edit} key under the configuration options in the \texttt{\_output.yml} file, which allows users to suggest an edit by linking to a chapter's source file.

Read more about the features of this output format here:

\url{https://pkgs.rstudio.com/bookdown/reference/gitbook.html}

Or use:

\begin{Shaded}
\begin{Highlighting}[]
\NormalTok{?bookdown}\SpecialCharTok{::}\NormalTok{gitbook}
\end{Highlighting}
\end{Shaded}

\chapter{Estatística Descritiva}\label{estatuxedstica-descritiva}

\section{Publishing}\label{publishing-2}

HTML books can be published online, see: \url{https://bookdown.org/yihui/bookdown/publishing.html}

\section{404 pages}\label{pages-2}

By default, users will be directed to a 404 page if they try to access a webpage that cannot be found. If you'd like to customize your 404 page instead of using the default, you may add either a \texttt{\_404.Rmd} or \texttt{\_404.md} file to your project root and use code and/or Markdown syntax.

\section{Metadata for sharing}\label{metadata-for-sharing-2}

Bookdown HTML books will provide HTML metadata for social sharing on platforms like Twitter, Facebook, and LinkedIn, using information you provide in the \texttt{index.Rmd} YAML. To setup, set the \texttt{url} for your book and the path to your \texttt{cover-image} file. Your book's \texttt{title} and \texttt{description} are also used.

This \texttt{gitbook} uses the same social sharing data across all chapters in your book- all links shared will look the same.

Specify your book's source repository on GitHub using the \texttt{edit} key under the configuration options in the \texttt{\_output.yml} file, which allows users to suggest an edit by linking to a chapter's source file.

Read more about the features of this output format here:

\url{https://pkgs.rstudio.com/bookdown/reference/gitbook.html}

Or use:

\begin{Shaded}
\begin{Highlighting}[]
\NormalTok{?bookdown}\SpecialCharTok{::}\NormalTok{gitbook}
\end{Highlighting}
\end{Shaded}

\chapter{Estatística - Medidas de Tendência Centrais e Dispersões}\label{estatuxedstica---medidas-de-tenduxeancia-centrais-e-dispersuxf5es}

\section{Publishing}\label{publishing-3}

HTML books can be published online, see: \url{https://bookdown.org/yihui/bookdown/publishing.html}

\section{404 pages}\label{pages-3}

By default, users will be directed to a 404 page if they try to access a webpage that cannot be found. If you'd like to customize your 404 page instead of using the default, you may add either a \texttt{\_404.Rmd} or \texttt{\_404.md} file to your project root and use code and/or Markdown syntax.

\section{Metadata for sharing}\label{metadata-for-sharing-3}

Bookdown HTML books will provide HTML metadata for social sharing on platforms like Twitter, Facebook, and LinkedIn, using information you provide in the \texttt{index.Rmd} YAML. To setup, set the \texttt{url} for your book and the path to your \texttt{cover-image} file. Your book's \texttt{title} and \texttt{description} are also used.

This \texttt{gitbook} uses the same social sharing data across all chapters in your book- all links shared will look the same.

Specify your book's source repository on GitHub using the \texttt{edit} key under the configuration options in the \texttt{\_output.yml} file, which allows users to suggest an edit by linking to a chapter's source file.

Read more about the features of this output format here:

\url{https://pkgs.rstudio.com/bookdown/reference/gitbook.html}

Or use:

\begin{Shaded}
\begin{Highlighting}[]
\NormalTok{?bookdown}\SpecialCharTok{::}\NormalTok{gitbook}
\end{Highlighting}
\end{Shaded}


  \bibliography{book.bib,packages.bib}

\end{document}
